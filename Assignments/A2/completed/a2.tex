\gdef\lecturenumber{a2}
\gdef\subsectioncounters{1}
\input preamble.tex

\newcommand{\blanks}[1]{\ensuremath{\dashuline{\hspace{#1}}}}

\newcommand{\sqplusoneexp}[1]{\LP \textvtt{SqPlusOne}~{#1}\RP}

\begin{document}
\date{due: Friday, 2022--02--11}
\Chapter{\vspace*{-3ex}Assignment 2}
\vspace*{-2ex}

If you wish, you may work in a group of up to 3 on this assignment.
Only one member needs to submit the assignment (but make sure at least one sof you does),
but \textbf{each} group member must \textbf{also} submit a brief statement,
listing all the group members and describing
the approximate division of labour between group members, including an estimate of how
many hours you spent.  For example, if you are in a group of 2, we need one assignment
plus two statements, submitted separately.

If you are working alone, you may write an estimate the time you spent, but it is not required
(and no separate statement is required).

\bigskip

{\noindent ID number(s)}: 20311734\\[1ex]
{\noindent Name(s)}: Zhimin Zhao


\section*{Late policy}

Assignments may be submitted up to 24 hours late with no penalty.


\section*{Scanning}

Submitting a legible scan is acceptable, but please ensure that the total file size is less than 30 MB.
It is preferable to combine pages into a single PDF file.


\clearpage
\section{Language extension}
\label{sec:lang-ext}

Consider the following language, defined by a grammar and a big-step evaluation judgment.
The big-step evaluation given is incomplete, in the informal sense that it has no rules saying how to evaluate
$\sqplusoneexp{e}$.
%
\begin{grammar}
  integers & $n$
  \\
  values & $v$ &\bnfas& $n$
  \\
  expressions & $e$ &\bnfas& 
      $
      n
      \bnfaltBRK
      \plusexp{e_1}{e_2}
      \bnfaltBRK
      \sqplusoneexp{e}
      $
\end{grammar}

\judgbox{e \down v}{expression $e$ evaluates to value $v$}
\vspace*{-1.2ex}
    \begin{mathpar}
      \Infer{eval-const}
          {}
          {n \down n}
      \and
      \Infer{eval-add}
          {
            e_1 \down n_1
            \\
            e_2 \down n_2
          }
          {\plusexp{e_1}{e_2} \down n_1 + n_2}
    \end{mathpar}

{\noindent \textbf{Question 1(a).}}
Roughly following the structure of eval-add, design a rule ``eval-sqplusone'' such that
evaluating $\sqplusoneexp{e}$ will evaluate $e$, square the result, and add one.  Similar to how we use the
standard mathematical notation for addition, $n_1 + n_2$, in eval-add, you may use the notation $n^2$ for the square of $n$.
\[
     \Infer{eval-sqplusone}
           {
              \hspace*{110pt}
              e \down n
              % comment out next line by inserting % at the beginning,
              %   then add your solution just before the `}'
              \hspace*{110pt}
           }
           {
             \sqplusoneexp{e} \down n^2+1
             % add your solution here (TeX doesn't care about line breaks UNLESS they are repeated)
             %
           \hspace*{20pt}
          }
\]

\clearpage

\section{Proof techniques}

These questions are not about complete proofs.
In some of the questions, the conjecture is not even true, or you
have not been given enough information to do a complete proof.
Instead, they ask you to make progress on several different proof attempts
by using a specific proof technique.

In all of these questions, the grammar of expressions is the \emph{extended}
grammar (\Sectionref{sec:lang-ext}), and the system of rules deriving $e \down v$ includes the \emph{three} rules eval-const, eval-add, eval-sqplusone.

\textbf{Question 2(a).}
Using the extended grammar of expressions (\Sectionref{sec:lang-ext}),
list the cases produced by \emph{case analysis on $e_2$}.
The cases must correspond to the grammar.
Please do not attempt to complete the proof.

\begin{conjecture*} ~\\
  For all expressions $e_1$, $e_2$ and $e_3$, \\
  if $\plusexp{e_1}{e_3} \down v$ and $\sqplusoneexp{e_2} \down v + 1$ \\
  then $v = 42$.
\end{conjecture*}
\begin{proof}
  Consider cases of $e$.

  \begin{itemize}
  \item Case $e = n$
  \item Case $e = \plusexp{e'}{e''}$
  \item Case $e = \sqplusoneexp{e'}$
  \end{itemize}

\vspace*{10ex}
\end{proof}


\textbf{Question 2(b).}
Suppose that we want to attempt to prove the above conjecture by induction on $e_3$.
Write the appropriate induction hypothesis.
\vspace*{1.2ex}

For all expressions $e_1$, $e_2$ and $e_3'$, such that $e_3'\prec e_3$, if $\plusexp{e_1}{e_3'} \down v$ and $\sqplusoneexp{e_2} \down v + 1$, then $v = 42$.

\vspace*{10ex}

\textbf{Question 2(c).}
In this question, your goal is to derive
\[
  \plusexp{\plusexp{e_{11}}{7}}{e_2} \down 7
\]
The following lines are given.  Use equations, arithmetic,
and the rules eval-const, eval-add to derive the goal.

\begin{llproof}
  \Pf{}{}{e_{11} \down n_{11}}  {Given}
  \Pf{}{}{e_{2} \down n_{2}}  {Given}
  \Pf{}{}{n_{11} = -1}  {Given}
  \Pf{}{}{n_{2} = 1}  {Given}
  \Pf{}{}{e_{11} \down -1}  {By arithmetic}
  \Pf{}{}{\plusexp{e_{11}}{7} \down -1 + 7}  {By eval-add}
  \Pf{}{}{\plusexp{e_{11}}{7} \down 6}  {By arithmetic}
  \Pf{}{}{e_{2} \down 1}  {By arithmetic}
  \Pf{}{}{\plusexp{\plusexp{e_{11}}{7}}{e_{2}} \down 6 + 1}  {By eval-add}
  \Pf{}{}{\plusexp{\plusexp{e_{11}}{7}}{e_{2}} \down 7}  {By arithmetic}
  \Pf{}{}{\hspace*{150pt}}  {}
\end{llproof}

\vspace*{20ex}


\textbf{Question 2(d).}
In this question, use \emph{inversion}: write down all the facts given by inverting
on rule eval-sqplusone.  (I can't give you a specific goal because what you get depends on your rule, eval-sqplusone.)

\begin{llproof}
  \Pf{}{}{e \down n}  {By assumption}
  \Pf{}{}{\sqplusoneexp{e} \down n'}  {By assumption}
  \Pf{}{}{\sqplusoneexp{e} \down n^2 + 1}  {By eval-sqplusone}
  \Pf{}{}{n'=n^2+1}  {By arithmetic}
  \Pf{}{}{n=\pm\sqrt{n' - 1}}  {By arithmetic}
  \Pf{}{}{e_1 \down \pm\sqrt{n_1 - 1}}  {By arithmetic}
  \Pf{}{}{\hspace*{150pt}}  {}
~\\
~\\
\end{llproof}



\clearpage

\section{Typing}

\begin{grammar}
  types & $A$ &\bnfas& $\Nat \bnfaltBRK \Int$
\end{grammar}

\judgbox{e : A}{expression $e$ has type $A$}
\begin{mathpar}
    \Infer{type-natconst}
        {n \geq 0}
        {n : \Nat}
    \and
    \Infer{type-const}
        {}
        {n : \Int}
    \\
    \Infer{type-sqplusone}
        {e : \Int
        }
        {\sqplusoneexp{e} : \Nat}
    \and
    \Infer{type-add}
        {e_1 : \Int
          \\
          e_2 : \Int
        }
        {\plusexp{e_1}{e_2} : \Int}
    \and
    \Infer{type-sub}
        {e : \Nat}
        {e : \Int}
\end{mathpar}

\medskip

Prove the following conjecture.
\textbf{Hint}: Use induction.

\begin{conjecture} ~\\
  For all expressions $e$,
  it is the case that $e : \Int$.
\end{conjecture}
\begin{proof} ~
  By induction on e.\\
  induction hypothesis: for all expressions $e'$ and $e'$, such that $e'\prec e$, it is the case that $e'$:int\\
  Consider cases of $e'$:
  
  \begin{itemize}
  \item Case $e = n$\\
  \begin{llproof}
    \Pf{}{}{e:int}{By arithmetic}
  \end{llproof}
  \item Case $e = \plusexp{e_1}{e_2}$\\
  \begin{llproof}
    \Pf{}{}{e_1:int}{By IH}
    \Pf{}{}{e_2:int}{By IH}
    \Pf{}{}{\plusexp{e_1}{e_2}:int}{By type-add}
    \Pf{}{}{e:int}{By arithmetic}
  \end{llproof}
  \item Case $e = \sqplusoneexp{e'}$\\
  \begin{llproof}
    \Pf{}{}{e':int}{By IH}
    \Pf{}{}{\sqplusoneexp{e'}:nat}{By type-sqplusone}
    \Pf{}{}{\sqplusoneexp{e'}:int}{By type-sub}
    \Pf{}{}{e:int}{By arithmetic}
  \end{llproof}
  \end{itemize}
  \vfill
\end{proof}


\clearpage

\section{Natural deduction}

A simplified system of natural deduction, omitting both kinds of quantification, 
is shown in \Figureref{fig:nd}.  This is the starting point for this question.

\begin{figure}[h]
\fontsz{10pt}{
\begin{grammar}
  atomic formulas
  &
  $P, Q$
  \\
  formulas
  &
  $A, B, C$
  &\bnfas&
  $P
  \bnfalt
  A \imp B
  \bnfalt
  A \AND B
  \bnfalt
  A \OR B
  \bnfalt
  \trueformula
  \bnfalt
  \falseformula
  \bnfalt
  \lnot A
  $
\vspace{-2ex}
\end{grammar}
}

\judgbox{A \True}{$A$ is true}
  \vspace*{-3ex}
\begin{mathpar}
  \Infer{$\imp$Intro\noteassume{\fighi{x}}}
       {
         \NDassume{\fighi{x}}{A \True}{
           B \True
         }
       }
       {(A \imp B) \True}
  \and
  \Infer{$\imp$Elim}
       {
         A \imp B \True
         \\
         A \True
       }
       {B \True}
  \vspace{-2ex}
  \\
  \Infer{$\trueformula$Intro}
       {
       }
       {
         \trueformula \True
       }
  ~~~\text{(no elim.\ for $\trueformula$)}
  ~~~~~~~~
  \text{(intro.\ for $\falseformula$: see $\lnot$Elim below)}
  ~~~~~
  \Infer{$\falseformula$Elim}
       {
         \falseformula \True
       }
       {
         C \True
       }
  \\
  \Infer{$\AND$Intro}
       {
         A \True
         \\
         B \True
       }
       {A \AND B \True}
  \and
  \Infer{$\AND$Elim1}
       {
         A \AND B \True
       }
       {A \True}
  ~~~~~
  \Infer{$\AND$Elim2}
       {
         A \AND B \True
       }
       {B \True}
%   \\
%   \Infer{$\forall$Intro\noteassume{\fighi{x}}}
%        {
%          \NDassume{\fighi{x}}{a:\Nat}{
%            B \True
%          }
%        }
%        {(\All{a:\Nat} B) \True}
%   ~~~~~
%   \Infer{$\forall$Elim}
%        {
%          (\All{a:\Nat} B) \True
%          \\
%          n : \Nat
%        }
%        {[n/a]B \True}
  \\
  \Infer{$\OR$Intro1}
       {
         A \True
       }
       {A \OR B \True}
  \and
  \Infer{$\OR$Intro2}
       {
         B \True
       }
       {A \OR B \True}  
  \and
  ~~
  \Infer{$\OR$Elim\noteassume{\fighi{x},\fighi{y}}}
       {
         A \OR B \True
         \\
         \NDassume{\fighi{x}}{A \True}
            {
              C \True
            }
         \\
         \NDassume{\fighi{y}}{B \True}
            {
              C \True
            }
       }
       {
         C \True
       }
  \\
%   \Infer{$\exists$Intro}
%        {
%          n : \Nat
%          \\{}
%          \big([n/a]B\big) \True
%        }
%        {(\Exists{a:\Nat} B) \True}
%   ~~~~~
%   \Infer{$\exists$Elim\noteassume{\fighi{x,y}}}
%        {
%          (\Exists{a:\Nat} B) \True
%          \\{}
%          \ndassume{\fighi{x}}{a:\Nat}{
%          \NDassume{\fighi{y}}{B \True}{
%            C \True
%          }}
%        }
%        {C \True}
%    \\
   \Infer{$\lnot$Intro\noteassume{x}}
        {\NDassume{x}{A \True}{
            \falseformula \True
          }
        }
        {(\lnot A) \True}
   \and
  \Infer{$\lnot$Elim}
      {
        A \True
        \\
        (\lnot A) \True
      }
      {\falseformula \True}
\vspace*{-3ex}
\end{mathpar}

\caption{Natural deduction, without quantifiers}
\label{fig:nd}
\end{figure}

% \clearpage

\newcommand{\gentzeniff}{\mathrel{\supset\subset}}

\textbf{Question 4(a).}
Derive
$
\big( (Q \AND P) \OR P \big) \imp (Q \OR P)
\True
$.

\begin{mathpar}
\Infer{$\imp$Intro\noteassume{x}}
    {
    \ndassume{\hspace*{-2ex}x}{((Q \AND P) \OR P) \True}{
        \Infer{$\OR$Intro2}{
            \ndassume{y}{(Q \AND P) \True}{
                \Infer{$\AND$Elim2}{
                    \Infer{y}{}{(Q \AND P) \True}
                }
                {P \True}
            }
        }
        {(Q \OR P) \True}
    }
    }
    {((Q \AND P) \OR P) \imp (Q \OR P) \True}
\end{mathpar}

\textbf{Question 4(b).}
Complete the following 
derivation of $(\lnot P) \imp (P \imp \falseformula) \True$.

\begin{mathpar}
    \Infer{$\imp$Intro\noteassume{x}}
      {
        \ndassume{\hspace*{-6ex}x}{\lnot P \True}{
          \Infer{$\imp$Intro\noteassume{y}}
             {
               \ndassume{y}{P \True}{
               \Infer{$\lnot$Elim}
               {
               \Infer{y}{}{P \True}
               \\
               \Infer{x}{}{\lnot P \True}
               }
               {\falseformula \True}
               }
             }
             {
               (P \imp \falseformula) \True
             }
           }
      }
      {(\lnot P) \imp (P \imp \falseformula) \True}
\end{mathpar}

\textbf{Question 4(c).}  Derive $(P \imp \falseformula) \imp (\lnot P) \True$.

\begin{mathpar}
\Infer{$\imp$Intro\noteassume{x}}
    {
    \ndassume{\hspace*{1ex}x}{(P \imp \falseformula) \True}{
        \Infer{$\lnot$Intro\noteassume{y}}{
            \ndassume{y}{P \True}{
                \Infer{$\imp$Elim}{
                    \Infer{x}{}{(P \imp \falseformula) \True}
                    \\
                    \Infer{y}{}{P \True}
                }
                {\falseformula \True}
            }
        }
        {(\lnot P) \True}
    }
    }
    {(P \imp \falseformula) \imp (\lnot P) \True}
\end{mathpar}

\textbf{Question 4(d).}  Derive $P \imp (\lnot \lnot P) \True$.

\begin{mathpar}
    \Infer{$\imp$Intro\noteassume{x}}
      {
        \ndassume{\hspace*{-9ex}x}{P \True}{
          \Infer{$\imp$Intro\noteassume{y}}
             {
               \ndassume{y}{\lnot P \True}{
               \Infer{$\lnot$Elim}
               {
               \Infer{x}{}{P \True}
               \\
               \Infer{y}{}{\lnot P \True}
               }
               {\falseformula \True}
               }
             }
             {
               (\lnot \lnot P) \True
             }
           }
      }
      {P \imp (\lnot \lnot P) \True}
\end{mathpar}

\vspace*{5ex}

\textbf{Question 4(e). (BONUS)}
Do we need negation?  What could we use instead?
\textbf{Hint}: Think about some of the previous questions.

No. We can use $P\imp\falseformula$ instead since $\imp$Intro simulates $\lnot$Intro while $\imp$Elim simulates $\lnot$Elim.

\begin{mathpar}
  \Infer{$\imp$Intro\noteassume{\fighi{x}}}
       {
         \NDassume{\fighi{x}}{P \True}{
           \falseformula \True
         }
       }
       {(P \imp \falseformula) \True}
  \and
  \Infer{$\imp$Elim}
       {
         (P \imp \falseformula) \True
         \\
         P \True
       }
       {\falseformula \True}
\end{mathpar}
\end{document}


% Local Variables:
% TeX-master: "a2"
% End:
