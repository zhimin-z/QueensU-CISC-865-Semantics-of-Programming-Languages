\gdef\lecturenumber{a3}
\gdef\subsectioncounters{1}
\input preamble.tex

\newcommand{\blanks}[1]{\ensuremath{\dashuline{\hspace{#1}}}}

\begin{document}
\date{due: Wednesday, 22 March 2022}
\Chapter{\vspace*{-2ex}Assignment 3}

If you wish, you may work in a group of up to 3 on this assignment.
Only one member needs to submit the assignment (but make sure at least one of you does),
but \textbf{each} group member must \textbf{also} submit a brief statement,
listing all the group members and describing
the approximate division of labour between group members, including an estimate of how
many hours you spent.  For example, if you are in a group of 2, we need one assignment
plus two statements, submitted separately.

If you are working alone, you may write an estimate the time you spent, but it is not required
(and no separate statement is required).

\bigskip

{\noindent ID number(s):} 20311734\\[1ex]

{\noindent Name(s) (optional):} Zhimin Zhao\\[1ex]


\section*{Late policy}

Assignments may be submitted up to 24 hours late with no penalty.


\section*{Scanning}

Submitting a legible scan is acceptable, but please ensure that the total file size is less than 30 MB.
It is preferable to combine pages into a single PDF file.

\vspace*{15ex}

\textbf{Note:} The question marked with a \textbf{+} is a bonus question,
and \textbf{it can only add up to 5\% to your mark}.
You can receive 100\% without doing it.

\clearpage


\section{Typing}

In this assignment, we consider the following
expressions $e$, types $T$ (or $S$), and typing contexts $\Gamma$:

$~$\hspace{-5ex}\begin{minipage}{0.4\linewidth}
\begin{grammar}
  % Expressions 
  & $e$
  &\bnfas&
  $
  \unitexp %\bnfaltBRK \recExp{x}{S}{e}
  \bnfaltBRK
  n \bnfalt \plusexp{e}{e} \bnfalt \minusexp{e}{e}
  \bnfaltBRK
  \trueexp \bnfalt \falseexp \bnfalt \iteexp{e}{e}{e}
  \bnfaltBRK
  \eqexp{e}{e} \bnfalt \ltexp{e}{e}
  \bnfaltBRK
  x \bnfalt \lamexp{x}{e} \bnfalt \callexp{e}{e}
  \bnfaltBRK
  \pairexp{e}{e} \bnfalt \projexp{1}{e} \bnfalt \projexp{2}{e}
  \bnfaltBRK
  \injexp{1}{e} \bnfalt \injexp{2}{e} \bnfaltBRK \caseexp{e}{x}{e}{x}{e}
  $
\end{grammar}
\end{minipage}
~
\begin{minipage}{0.22\linewidth}
\begin{grammar}
% Types
&  $S, T$
&\bnfas&
$
\unitty
\bnfaltBRKn{unit type}
\Int
\bnfaltBRKn{type of integers}
\Bool
\bnfaltBRKn{type of booleans}
S -> T
\bnfaltBRKn{type of functions on $S$ that produce $T$}
S ** T
\bnfaltBRKn{type of pairs of one $S$ and one $T$}
S + T
% \BRKn{\emph{disjoint union} or \emph{sum} type: contains either an $S$ or a $T$}
\BRKn{type of a left-injected $S$ or right-injected $T$}
$
\\[1ex]
% Contexts
& $\Gamma$
&\bnfas&
$
\emptyctx
\bnfaltBRKn{empty context}
\Gamma, x : S
\BRKn{$x$ has type $S$}
$
\end{grammar}
\end{minipage}

\vspace{1em}

Consider the following typing rules given in \Figureref{fig:typing}.

\begin{figure}[th]
\judgbox{\Gamma |- e : T}{Under assumptions $\Gamma$, expression $e$ has type $T$}
  \vspace*{-1ex}
\begin{mathpar}
  \Infer{type-assum}
       { (x : S) \in \Gamma
       }
       {
         \Gamma |- x : S
       }
  \and
  \Infer{$->$Intro}
       {
         \Gamma, x : S
         |-
         e : T
       }
       {\Gamma |- \lamexp{x}{e} : (S -> T)}
  \and
  \Infer{$->$Elim}
       {
         \Gamma |- e_1 : (S -> T)
         \\
         \Gamma |- e_2 : S
       }
       {\Gamma |- \callexp{e_1}{e_2} : T}
  \vspace{-1.5ex}
  \\
  \Infer{$\unitty$Intro}
       {}
       {\Gamma |- \unitexp : \unitty}
  ~~~~~~~~~~~~~~
  \Infer{type-true}
       {}
       {\Gamma |- \Btrue : \Bool}
  \and
  \Infer{type-false}
       {}
       {\Gamma |- \Bfalse : \Bool}
  \\
  \Infer{$\Int$Intro}
       {
       }
       {\Gamma |- n : \Int}
  \and
  \Infer{type-add}
       {
         \Gamma |- e_1 : \Int
         \\
         \Gamma |- e_2 : \Int
       }
       {\Gamma |- \plusexp{e_1}{e_2} : \Int}
  \and
  \Infer{type-sub}
       {
         \Gamma |- e_1 : \Int
         \\
         \Gamma |- e_2 : \Int
       }
       {\Gamma |- \minusexp{e_1}{e_2} : \Int}
  \\
  \Infer{type-equals}
       {
         \Gamma |- e_1 : \Int
         \\
         \Gamma |- e_2 : \Int
       }
       {\Gamma |- \eqexp{e_1}{e_2} : \Bool}
  \and
  \Infer{type-lt}
       {
         \Gamma |- e_1 : \Int
         \\
         \Gamma |- e_2 : \Int
       }
       {\Gamma |- \ltexp{e_1}{e_2} : \Bool}
  \\
  \Infer{type-ite}
       {
         \Gamma |- e : \Bool
         \\
         \Gamma |- \eThen : T
         \\
         \Gamma |- \eElse : T
       }
       {\Gamma |- \iteexp{e}{\eThen}{\eElse} : T}
  \\
  \Infer{$+$Intro1}
       {
         \Gamma |- e_1 : S_1
       }
       {\Gamma |- \injexp{1}{e_1} : (S_1 + S_2)}
  ~~~~~
  \Infer{$+$Intro2}
       {
         \Gamma |- e_2 : S_2
       }
       {\Gamma |- \injexp{2}{e_2} : (S_1 + S_2)}
  \and
  \Infer{$+$Elim}
       {
         \Gamma |- e : (S_1 + S_2)
         \\
         \Gamma, x_1 : S_1
         |-
         e_1 : T
         \\
         \Gamma, x_2 : S_2
         |-
         e_2 : T
       }
       {
         \Gamma |- \caseexp{e}{x_1}{e_1}{x_2}{e_2} : T
       }
   \\
  \Infer{$**$Intro}
       {
         \Gamma |- e_1 : S_1
         \\
         \Gamma |- e_2 : S_2
       }
       {\Gamma |- \pairexp{e_1}{e_2} : (S_1 ** S_2)}
  \and
  \Infer{$**$Elim1}
       {
         \Gamma |- e : (S_1 ** S_2)
       }
       {\Gamma |- \projexp{1}{e} : S_1}
  \and
  \Infer{$**$Elim2}
       {
         \Gamma |- e : (S_1 ** S_2)
       }
       {\Gamma |- \projexp{2}{e} : S_2}
\vspace*{-2.3ex}
\end{mathpar}

\caption{Typing with functions ($->$), integers, booleans, sums ($+$), and pairs ($\times$)}
\label{fig:typing}
\end{figure}

\clearpage

\textbf{Question 1(a).}
Let $\Gamma = (x : \Bool)$.
Complete the following typing derivation.
%
\vspace*{10ex}
\[
  \Infer{type-ite}
        {
          \Infer{\!type-assum}
                 {(x : \Bool) \in \Gamma}
                 {\Gamma |- x : \Bool}
          ~~~~~
          \Infer{\!intIntro}{}
                 {\Gamma |- 2 : \Int}
          ~~~~~
          \Infer{\!intIntro}{}
                 {\Gamma |- 8 : \Int}
        }
        {\Gamma |- \iteexp{x}{2}{8} : \Int}
\]

\textbf{Question 1(b).}
Let $\Gamma = (x : (\Int + \Bool))$.
% \\
Complete the following typing derivation.
%
\vspace*{14ex}
\[
\hspace*{-6ex}
  \Infer{+Elim}
        {
          \Infer{\!type-assum}
                 {(x : (\Int + \Bool)) \in \Gamma}
                 {\Gamma |- x : (\Int + \Bool)}
          ~~~~~
          \Infer{\!type-false}
                 {}
                 {\Gamma, z : \Int |- \Bfalse : \Bool}
          ~~~~~
          \Infer{\!type-assum}
                 {(y : \Bool) \in \Gamma}
                 {\Gamma, y : \Bool |- y : \Bool}
        }
        {\Gamma |- \caseexp{x}{z}{\Bfalse}{y}{y} : \Bool}
\]

\textbf{Question 1(c).} Booleans are not really necessary,
because we can write $\injexp{1}{\unitexp}$ instead of $\Btrue$,
$\injexp{2}{\unitexp}$ instead of $\Bfalse$, and \textvtt{Case}
instead of \textvtt{Ite}.
Translate the expression from 1(a), $\iteexp{x}{2}{8}$,
into an expression that has type $\Int$ under the typing context
\[
      x : (\unitty + \unitty)
\]
and which would step to $2$ if $x$ were replaced with $\injexp{1}{\unitexp}$,
and to $8$ if $x$ were replaced with $\injexp{2}{\unitexp}$.
\textbf{Hint:} think about the derivation in 1(b).
    
\[
\caseexp{x}{x_1}{2}{x_2}{8}
\]

\clearpage
\section{Mirror World}

Consider the sequent calculus rules in \Figureref{fig:sc}.

\begin{figure}[h]
\judgbox{\Gamma |- A \True}{Under assumptions $\Gamma$, formula $A$ is true}
  \vspace*{-1ex}
\begin{mathpar}
  \Infer{sc-assum}
       { \xNDassume{x}{A \True} \in \Gamma
       }
       {
         \Gamma |- A \True
       }
  \and
  \Infer{sc-$\imp$Intro}
       {
         \Gamma, \xNDassume{x}{A \True}
         |-
         B \True
       }
       {\Gamma |- (A \imp B) \True}
  \and
  \Infer{sc-$\imp$Elim}
       {
         \Gamma |- A \imp B \True
         \\
         \Gamma |- A \True
       }
       {\Gamma |- B \True}
  \vspace{-1.5ex}
  \\
  \Infer{sc-$\trueformula$Intro}
       {}
       {\Gamma |- \trueformula \True}
  \\
  \Infer{sc-$\OR$Intro1}
       {
         \Gamma |- A \True
       }
       {\Gamma |- A \OR B \True}
  ~~
  \and
  \Infer{sc-$\OR$Intro2}
       {
         \Gamma |- B \True
       }
       {\Gamma |- A \OR B \True}  
  ~~~
  \and
  \Infer{sc-$\OR$Elim}
       {
         \Gamma |- A \OR B \True
         \\
         \Gamma, \xNDassume{x}{A \True}
         |-
         C \True
         \\
         \Gamma, \xNDassume{y}{B \True}
         |-
         C \True
       }
       {
         \Gamma |- C \True
       }
   \\
  \Infer{sc-$\AND$Intro}
       {
         \Gamma |- A \True
         \\
         \Gamma |- B \True
       }
       {\Gamma |- A \AND B \True}
  \and
  \Infer{sc-$\AND$Elim1}
       {
         \Gamma |- A \AND B \True
       }
       {\Gamma |- A \True}  
  \and
  \Infer{sc-$\AND$Elim2}
       {
         \Gamma |- A \AND B \True
       }
       {\Gamma |- B \True}  
\vspace*{-2.3ex}
\end{mathpar}

\caption{Sequent calculus, with $\imp$, $\trueformula$, $\OR$, and $\AND$}
\label{fig:sc}
\end{figure}

Some of the typing rules from \Figureref{fig:typing}
have a curious property:
if we
change some of the meta-variables,
remove the expression and colon,
translate the types,
and add the word $\xTrue$, we get a rule in \Figureref{fig:sc}.

For example, the rule $\times$Elim1
becomes the rule sc-$\AND$Elim1:
\[\runonfontsz{9.7pt}
   \Infer{}%{$\times$Elim1}
          {\Gamma |- e : S_1 ** S_2}
          {\Gamma |- \Proj{1} e : S_1}
   ~~\Longrightarrow~~
   \Infer{}
          {\Gamma |- e : A_1 ** A_2}
          {\Gamma |- \Proj{1} e : A_1}
   ~~\Longrightarrow~~
   \Infer{}
          {\Gamma |- A_1 ** A_2}
          {\Gamma |- A_1}
   ~~\Longrightarrow~~
   \Infer{}
          {\Gamma |- A_1 \AND A_2}
          {\Gamma |- A_1}
   ~~\Longrightarrow~~
   \Infer{}
          {\Gamma |- A_1 \AND A_2 \True}
          {\Gamma |- A_1 \True}
\]
Here we translated the type $S_1 ** S_2$
to $A_1 \AND A_2$ by replacing $**$ with $\AND$.
The types can be translated as follows:
\[
\begin{tabular}[t]{rcll}
  $\unitty$ & becomes & $\trueformula$
\\
  $**$ & becomes & $\AND$
\\
  $+$ & becomes & $\OR$
\\
  $->$ & becomes & $\imp$
\end{tabular}
\]
Assumptions need some extra work;
for example, in translating $->$Intro,
$x : S$ becomes $\xNDassume{x}{S \True}$:

\begin{mathpar}
\runonfontsz{9.7pt}
   \Infer{}%{$\times$Elim1}
          {\Gamma, x : S |- e : T}
          {\Gamma |- \lamexp{x}{e} : S -> T}
   ~~\Longrightarrow~~
   \Infer{}
          {\Gamma, x : A |- e : B}
          {\Gamma |- \lamexp{x}{e} : A -> B}
   ~~\Longrightarrow~~
   \Infer{sc-$\imp$Intro}
          {\Gamma, \xNDassume{x}{A \True} |- B \True}
          {\Gamma |- (A \imp B) \True}
\end{mathpar}

Not all the typing rules in \Figureref{fig:typing} have meaningful translations.
The rules involving arithmetic, among others, do not lead to anything interesting.
\clearpage

\textbf{Question 2(a).}
Pick exactly three rules from \Figureref{fig:typing} such that
each has a meaningful translation (following the procedure above)
to exactly one corresponding rule in \Figureref{fig:sc}.
Do \textbf{not} pick $\times$Elim1 or $->$Intro.
In the table below, fill in rows (1), (2) and (3).
In the first column of each row, write a rule you chose;
and, in the second column, write its corresponding translation.

\vspace*{2ex}

\begin{tabular}{|l|l|}
  \hline
  \Figureref{fig:typing} rule\hspace{5em} & \Figureref{fig:sc} rule\hspace{5em} \\
  \hline
  ~~~~~~ $->$Intro & sc-$\imp$Intro \\[1ex]
  ~~~~~~ $\times$Elim1 & sc-$\AND$Elim1 \\[1ex]
  %%% Replace the tildes (~) below with your answers:
  (1) type-assum & sc-assum
  \\[1ex]
  (2) type-true & $\unitty$Intro 
  \\[1ex]
   (3) $->$Elim  & sc-$\imp$Elim \\[1ex]
  \hline
\end{tabular}

%Curry-Howard correspondence:
%- types S correspond to logical formulas A
%- expression e correspond to proof of formulas (latter corresponds to sequent calculus derivation: Γ ⊢ A true)

% Explanation for 2(a)(2)
%⟦unit⟧ = {⋆}
%⟦bool⟧ = {true, false}
%True has one normal proof. If bool corresponds to True, then True would have two normal proofs. Thus, bool does not correspond to True.
%\begin{mathpar}
%\runonfontsz{9.7pt}
%  \Infer{$\unitty$Intro}{}
%       {\Gamma |- \unitexp : \unitty}
%   ~~\Longrightarrow~~
%  \Infer{}{}
%       {\Gamma |- \unitexp : \True}
%   ~~\Longrightarrow~~
%  \Infer{}{}
%       {\Gamma |- \True}
%   ~~\Longrightarrow~~
%  \Infer{sc-$\trueformula$Intro}{}
%       {\Gamma |- \trueformula \True}
%\end{mathpar}

\vspace*{2ex}

\textbf{Question 2(b)+ (up to 5\% bonus).}

\textbf{A single $\AND$-elimination:}
We have two elimination rules for $\AND$, which separately extract
a subformula.  Design a \emph{single} sequent-calculus
elimination rule for $\AND$.
\textbf{Hint:} think about the structure of \text{sc-$\OR$Elim}.
\textbf{Second hint:} think about the connection between
$(P_1 \AND P_2) \imp Q$ and $P_1 \imp (P_2 \imp Q)$,
and the shape of a derivation of $\emptyctx |- P_1 \imp (P_2 \imp Q) \True$:
how many assumptions are needed within that derivation?

\begin{mathpar}
  \Infer{sc-$\AND$Elim}
       { 
         \Gamma |- (A \AND B) \True
         \\
         \Gamma, \xNDassume{x}{A \True}, \xNDassume{y}{B \True} |- C \True
       }
       {
         \Gamma |- C \True
       }
\end{mathpar}

\textbf{A single $\times$-elimination rule:}
Translate your new $\AND$-elimination rule into
a typing rule. You will need to extend the grammar
of expressions $e$.

$e \bnfas ... \bnfalt (\textvtt{Split} \; e (<x, x> => e))$
OR
$e \bnfas ... \bnfalt (\textvtt{Split} \; e (<x_1, x_2> => e'))$

\begin{mathpar}
  \Infer{sc-$\times$Elim}
       { 
         \Gamma |- e : (S_1 \times S_2)
         \\
         \Gamma, x_1 : S_1, x_2 : S_2 |- e' : T
       }
       {
         \Gamma |- \textvtt{Split} \; e (<x_1, x_2> => e') : T
       }
\end{mathpar}

\textbf{Small-step semantics:}
Extend the small-step semantics (see next page) with a reduction rule
for your new expression form, and extend the grammar $\C$
of contexts as appropriate.

$C \bnfas ... \bnfalt (\textvtt{Split} \; C (<x, x> => e))$

\begin{mathpar}
  \Infer{\!red-split}{}
       {
         \textvtt{Split} \; \pairexp{v_1}{v_2} (<x_1, x_2> => e) 
         \stepR
         [v_1/x_1, v_2/x_2]e
       }
\end{mathpar}

\clearpage
\section{Progress}

For this question, we consider the following small-step semantics.

\begin{minipage}{0.4\linewidth}
\begin{grammar}
  Values & $v$
  &\bnfas&
  $
  \unitexp
  \bnfaltBRK
  n
  \bnfaltBRK
  \trueexp \bnfalt \falseexp
  \bnfaltBRK
  x
  \bnfalt
  \lamexp{x}{e}
  \bnfaltBRK
  \pairexp{v}{v}
  \bnfaltBRK
  \injexp{1}{v} \bnfalt \injexp{2}{v}
  $
\end{grammar}
\end{minipage}
~
\begin{minipage}{0.22\linewidth}
\begin{grammar}
  Evaluation contexts
  & $\C$ &\bnfas&
  $
  \hole
  \bnfaltBRK  \plusexp{\C}{e}  \bnfalt  \plusexp{v}{\C}
  \bnfaltBRK  \minusexp{\C}{e}  \bnfalt  \minusexp{v}{\C}
  \bnfaltBRK  \ite{\C}{e}{e}
  \bnfaltBRK  \eqexp{\C}{e}  \bnfalt  \eqexp{v}{\C}
  \bnfaltBRK  \ltexp{\C}{e}  \bnfalt  \ltexp{v}{\C}
  \bnfaltBRK  \callexp{\C}{e}
  \bnfaltBRK  \callexp{v}{\C}
  \bnfaltBRK  \pairexp{\C}{e} \bnfalt \pairexp{v}{\C}
  \bnfaltBRK  \projexp{1}{\C} \bnfalt \projexp{2}{\C}
  \bnfaltBRK  \injexp{1}{\C} \bnfalt \injexp{2}{\C} \bnfaltBRK \caseexp{\C}{x}{e}{x}{e}
  $
\end{grammar}
\end{minipage}

\bigskip

\judgbox{e \stepR e'}{Expression $e$ reduces to $e'$}
\vspace*{-1.5ex}
\begin{mathpar}
  \Infer{red-add}
     {
     }
     {\plusexp{n_1}{n_2} \stepR (n_1 + n_2)}
  \and
  \Infer{red-sub}
     {
     }
     {\minusexp{n_1}{n_2} \stepR (n_1 - n_2)}
  \vspace*{-1.5ex}
  \\
  \Infer{\!red-equals}
     {
     }
     {\eqexp{n_1}{n_2} \stepR (n_1 = n_2)}
  \and
  \Infer{\!red-lessthan}
     {}
     {\ltexp{n_1}{n_2} \stepR (n_1 < n_2)}
  \vspace*{-1.5ex}
  \\
  \Infer{\!red-ite-then}
     {}
     {\iteexp{\trueexp}{\eThen}{\eElse}  \stepR   \eThen}
  \and
  \Infer{\!red-ite-then}
     {}
     {\iteexp{\trueexp}{\eThen}{\eElse}  \stepR   \eThen}
  \\
  \Infer{\!red-beta}
     {}
     {\callexp{\lamexp{x}{e}}{v}  \stepR   [v/x]e}     
  \vspace*{-1.0ex}
  \\
  \Infer{\!red-proj1}
     {}
     {\projexp{1}{\pairexp{v_1}{v_2}} \stepR v_1}
  \and
  \Infer{\!red-proj2}
     {}
     {\projexp{2}{\pairexp{v_1}{v_2}} \stepR v_2}
  \vspace*{-1ex}
  \\
  \Infer{\!red-case1}
     {}
     {
       \caseexp{\injexp{1}{v_1}}{x_1}{e_1}{x_2}{e_2}
       \stepR
       [v_1/x_1]e_1
     }
  \vspace*{-2ex}
  \\
  \Infer{\!red-case2}
     {}
     {
       \caseexp{\injexp{2}{v_2}}{x_1}{e_1}{x_2}{e_2}
       \stepR
       [v_2/x_2]e_2
     }
  % \\
  % \Infer{\!red-rec}
  %     {}
  %     {
  %       \recExp{f}{S}{e}
  %       ~\stepR~
  %       \big[\recExp{f}{S}{e} \,\big/\, f\big]e
  %     }
\end{mathpar}

\bigskip

\judgbox{e \step e'}{Expression $e$ takes one step to $e'$}
\begin{mathpar}
     \Infer{step-context}
         {e \stepR e'}
         {\C[e] \step \C[e']}
\end{mathpar}

\clearpage

\textbf{Question 3(a).}
\emph{Progress} says that
\[
\text{
  If $\emptyctx |- e : S$ then either (1) $e$ is a value, or (2) there exists
  $e'$ such that $e \step e'$.
}
\]
For most languages, including ours, it is impossible to prove progress
without first proving a lemma known as \emph{canonical forms}
or \emph{value inversion}.

The first name, canonical forms, comes from the idea that the values of a given type---as opposed
to expressions that are not values---are the original or canonical forms of that type.
For example, while $\plusexp{1}{1}$ and $\minusexp{5}{3}$
are both \emph{expressions} of type
$\Int$---and, in a sense, represent the same integer $2$ since they all eventually step
to $2$---we would not consider these expressions as defining the set of integers.
But we can say that the \emph{values} of type $\Int$---which are the integer constants $n$---define the integers.

The second name, value inversion, comes from the fact that the lemma
uses inversion on a given derivation---but not the inversion we have often used,
where we reason either from (a) knowing that we have an expression $e$
of a particular form, say $\callexp{e_1}{e_2}$, or (b) knowing that the conclusion
of a derivation is by some particular rule, say $->$Elim.
Instead, the inversion is based on the combination of two facts:

\begin{itemize}
\item We know that the expression is a value.
\item We know something about the expression's type.
\end{itemize}

Here is the complete value inversion, or canonical forms, lemma for our current language.
There is one part for each production in the grammar of types.

\begin{lemma}[Value Inversion] \label{lem:value-inversion}
    ~
  \begin{enumerate}
  \item If $\emptyctx |- v : \unitty$
    then $v = \unitexp$.

  \item If $\emptyctx |- v : \Bool$
    then either $v = \Btrue$ or $v = \Bfalse$.

  \item If $\emptyctx |- v : \Int$
    then there exists $n$ such that $v = n$.
    
  \item If $\emptyctx |- v : (S_1 ** S_2)$
    \\
    then there exist $v_1$ and $v_2$ such that
    $v = \pairexp{v_1}{v_2}$
    and $\emptyctx |- v_1 : S_1$
    and $\emptyctx |- v_2 : S_2$.

  \item If $\emptyctx |- v : (S_1 -> S_2)$
    then there exist $x$ and $e$ such that
    $v = \lamexp{x}{e}$
    and $x : S_1 |- e : S_2$.

  \item If $\emptyctx |- v : (S_1 + S_2)$
    then either (1) there exists $v_1$ such that $v = \injexp{1}{v_1}$
    and $\emptyctx |- v_1 : S_1$
    or (2) there exists $v_2$ such that $v = \injexp{2}{v_2}$
    and $\emptyctx |- v_2 : S_2$.
  \end{enumerate}
\end{lemma}


\begin{proof}{}
  [Unusually, this proof does not need induction.]

  Part 1: The only rule whose conclusion can be $\emptyctx |- \unitexp : \unitty$
  is $\unitty$Intro.  By inversion on $\unitty$Intro, we have $v = \unitexp$.
  [In full detail for the impossible cases:
  
  \begin{itemize}
  \item In type-assum, we have $x$ (which is a value)
    but the context is $\emptyctx$, so the premise is $(x : \unitty) \in \emptyctx$
    which is impossible.

  \item In $->$Intro, the expression being typed is a value,
    but the type is a $->$ which does not match the given $\unitty$,
    so this case is impossible.

  \item In $->$Elim, the expression being typed has the form $\callexp{e_1}{e_2}$,
    which is not a value.

  \item In type-true, type-false and $\Int$Intro, the expression being typed
    is a value but the type does not match.

  \item In type-add, type-sub, type-abs, type-equals, type-lt, type-ite, $**$Elim1 and
    $**$Elim2, % and type-rec,
    the expression being typed is not a value.

  \item In $**$Intro, the expression being typed could be a value (if the
    two subexpressions $e_1$ and $e_2$ are values, then $\pairexp{e_1}{e_2}$
    is a value), but the type does not match.
  \end{itemize}
  End of the detail for the impossible cases.]

  \medskip

  Parts 2, 3, 4, 5: [proof omitted]

  \medskip

  Part 6: \textbf{Question 3(a).} Fill in the four listed cases.

  Consider cases of the rule concluding $\emptyctx |- v : (S_1 + S_2)$.
  [Either explain why the case is impossible, \emph{even if you are only
    repeating what I gave for Part 1}, or prove the goal for Part 6:\\
``either (1) there exists $v_1$ such that $v = \injexp{1}{v_1}$
    and $\emptyctx |- v_1 : S_1$
    or (2) there exists $v_2$ such that $v = \injexp{2}{v_2}$
    and $\emptyctx |- v_2 : S_2$.''

  \begin{itemize}
  \item type-assum:
  
  \begin{llproof}
  \Pf{}{}{\emptyctx |- v : (S_1 + S_2)}{Given}
  \Pf{}{}{\Gamma |- x : S}{By inversion}
  \Pf{}{}{\emptyctx |- x : (S_1 + S_2)}{By above equation}
  \end{llproof}
  
  However, the premise then becomes $x : (S_1 + S_2) \in \emptyctx$, which is impossible.

  \item $->$Intro:
  
  \begin{llproof}
  \Pf{}{}{\emptyctx |- v : (S_1 + S_2)}{Given}
  \Pf{}{}{\Gamma |- (Lam x e) : (S -> T)}{By inversion}
  \end{llproof}
  
  However, by above equation we have $(S_1 + S_2) = (S -> T)$, which is impossible.

  \item $+$Elim:
  
  \begin{llproof}
  \Pf{}{}{\emptyctx |- v : (S_1 + S_2)}{Given}
  \Pf{}{}{\Gamma |- \caseexp{e}{x_1}{e_1}{x_2}{e_2} : T}{By inversion}
  \Pf{}{}{\caseexp{e}{x_1}{e_1}{x_2}{e_2} = v}{By above equation}
  \end{llproof}
  
  However, the expression has the form $\caseexp{e}{x_1}{e_1}{x_2}{e_2}$, which is not possibly a value.

  \item $+$Intro1:
  
  \begin{llproof}
  \Pf{}{}{\emptyctx |- v : (S_1 + S_2)}{Given}
  \Pf{}{}{v = \injexp{1}{e_1}}{By inversion on $+$Intro1}
  \Pf{}{}{e_1 = v_1}{By inversion on value grammar}
  \Pf{}{}{v = \injexp{1}{v_1}}{By above equations}
  \Pf{}{}{\emptyctx |- v_1 : S_1}{By inversion on $+$Intro1}
  \end{llproof}

  \item $+$Intro2: Similar to the $+$Intro1 case.

  \item{} The remaining cases are impossible for reasons similar to those
    given in Part 1.
  \qedhere
  \end{itemize}


\end{proof}


\end{document}

% Local Variables:
% TeX-master: "a3"
% End:
