\synctex=1

\gdef\lecturenumber{Midterm}
\gdef\subsectioncounters{1}
\input preamble.tex
\usepackage{etoolbox}

\newcommand{\blanks}[1]{\ensuremath{\dashuline{\hspace{#1}}}}


\begin{document}
\date{Friday, 2022--03--04, 12:30 p.m. to Saturday, 2022--03--05, 12:30 p.m.}


\Chapter{\vspace*{-4ex}CISC 465 take-home midterm}

There is a 24-hour grace period, so the actual deadline is \textbf{Sunday} at 12:30 p.m.

\bigskip

\textbf{Student ID number:}  \uline{\hspace*{0.7\textwidth}}

\medskip

Please note: You must \textbf{NOT} work in a group on the midterm.


\section*{Total points: 60}

\section{Q1: Grammars, Inductive Definitions and Induction (15 points)}

Consider the following grammar:

\newcommand{\lcabs}[2]{\texttt(\lambda{#1}.\;{#2}\texttt)}
\newcommand{\lcapp}[2]{{#1}\texttt(#2\texttt)}

\begin{grammar}
integers & $n$ &
\\
variables & $x, y, z$ &
\\
terms  & $M, N$ &\bnfas&
      $x
      \bnfaltBRKn{}
      n
      \bnfaltBRKn{}
      \lcabs{x}{N}
      \bnfaltBRKn{}
      \lcapp{M}{M}
      \BRKn{}
      $
\end{grammar}

\textbf{Question 1(a).} This grammar corresponds to an inductive definition.  Fill in the missing parts of the definition.

\begin{enumerate}
\item If $x$ is a variable then $x$ is a term.

\item
~\\[5ex]

\item
~\\[5ex]

\item
~\\[5ex]

\end{enumerate}

\textbf{Question 1(b).} For each string of symbols on the left, determine whether it is, or is not, a term (according to the above grammar).

\medskip

\begin{tabular}{rcll}
&  $\texttt(\lambda x.\;z\texttt)$ &  is a term~~~~~is not a term
\\[1ex]
&  $\texttt(\lambda y, y.\;y\texttt)$ & is a term~~~~~is not a term
\\[1ex]
&  $\texttt(\lambda x.\;x\texttt(3\texttt)\texttt)$ &  is a term~~~~~is not a term
\\[1ex]
&  $3\texttt(x\texttt)$ & is a term~~~~~is not a term
\end{tabular}

\clearpage

A term \emph{mentions} a variable if that variable appears somewhere in the term.
For example:

\begin{itemize}
\item  $y$ mentions $y$
\item  $\lcabs{x}{y}$ mentions $x$ and $y$
\item  $\lcabs{z}{z}$ mentions $z$
\item  $\lcabs{x}{1}$ mentions $x$
\end{itemize}


\textbf{Question 1(c).}
For the following conjecture, write the induction hypothesis and the appropriate set
of cases to consider.  You do \textbf{not} need to complete any of the cases (as it happens, the conjecture is false).

\begin{conjecture*}
  For every term $M$, there exists a variable $x$ such that $M$ mentions $x$.
\end{conjecture*}
\begin{proof}
  By structural induction on $M$.

  \medskip

  \textbf{Induction hypothesis}:

  \vspace*{10ex}


  Consider cases of $M$.

  \begin{itemize}
  \item Case



  \vspace*{33ex}
  \end{itemize}
\end{proof}





\clearpage


\section{Q2: Natural deduction and sequent calculus (20 points)}

In this question, we consider simplified versions of natural deduction
(lectures 5 through 8) and sequent calculus (lecture 9).

\begin{grammar}
  atomic formulas
  &
  $P, Q$
  \\
  formulas
  &
  $A, B, C$
  &\bnfas&
  $P
  \bnfaltBRKn{atomic formula}
  A \imp B
  \bnfaltBRKn{implication}
  A \AND B
  \BRKn{conjunction (and)}
  $
  \\[1ex]
  contexts (in sequent calculus)
  &
  $\Gamma$
  &\bnfas&
  $\emptyctx \bnfaltBRKn{empty context (no assumptions)}
  \Gamma, \xNDassume{x}{A \True}
  \BRKn{assumption that $A$ is true}
  $
\end{grammar}

The questions in this section involve both natural deduction
and sequent calculus rules.  To distinguish them, we prefix the
natural deduction rules with ``nat'' and the sequent calculus rules
with ``seq''.  For example, the rule called $\imp$Elim in lec5-7
is called nat-$\imp$Elim here, and the rule called $\imp$Elim in lec9
is called seq-$\imp$Elim here.

\bigskip

\judgbox{A \True}{$A$ is true (in natural deduction)}
  \vspace*{-1ex}
\begin{mathpar}
  \Infer{nat-$\imp$Intro\noteassume{\fighi{x}}}
       {
         \NDassume{\fighi{x}}{A \True}{
           B \True
         }
       }
       {(A \imp B) \True}
  \and
  \Infer{nat-$\imp$Elim}
       {
         A \imp B \True
         \\
         A \True
       }
       {B \True}
  \\
  \Infer{nat-$\AND$Intro}
       {
         A \True
         \\
         B \True
       }
       {A \AND B \True}
  \and
  \Infer{nat-$\AND$Elim1}
       {
         A \AND B \True
       }
       {A \True}
  ~~~~~
  \Infer{nat-$\AND$Elim2}
       {
         A \AND B \True
       }
       {B \True}
\end{mathpar}

Now, we give the corresponding sequent calculus rules:

\bigskip

\judgbox{\Gamma |- A \True}{Under assumptions $\Gamma$, formula $A$ is true (sequent calculus)}
  \vspace*{-1ex}
\begin{mathpar}
  \Infer{seq-assum}
       { \xNDassume{x}{A \True} \in \Gamma
       }
       {
         \Gamma |- A \True
       }
  \\
  \Infer{seq-$\imp$Intro}
       {
         \Gamma, \xNDassume{x}{A \True}
         |-
         B \True
       }
       {\Gamma |- (A \imp B) \True}
  \and
  \Infer{seq-$\imp$Elim}
       {
         \Gamma |- A \imp B \True
         \\
         \Gamma |- A \True
       }
       {\Gamma |- B \True}
  \\
  \Infer{seq-$\AND$Intro}
       {
         \Gamma |- A \True
         \\
         \Gamma |- B \True
       }
       {
         \Gamma |- 
         A \AND B \True
       }
  \and
  \Infer{seq-$\AND$Elim1}
       {
         \Gamma |- A \AND B \True
       }
       {
         \Gamma |- 
         A \True
       }
  ~~~~~
  \Infer{seq-$\AND$Elim2}
       {
         \Gamma |- A \AND B \True
       }
       {
         \Gamma |- 
         B \True
       }
\end{mathpar}

\clearpage



\textbf{Question 2(a).}
Complete the two derivations below.

In the second (sequent calculus) one, you may write
$\Gamma_{xy}$ instead of
\[
    \emptyset, \xNDassume{x}{Q \AND P \True},
    \xNDassume{y}{Q \True}
\]
Hint: When you use seq-assum, its premise should be
$\xNDassume{x}{Q \AND P \True} \in \Gamma_{xy}$.

\begin{minipage}{0.48\linewidth}
\vspace{40ex}
\[
\Infer{nat-$\imp$Intro\noteassume{\fighi{x}}}
   {
   }
   {(Q \AND P) \imp (Q \imp P) \True}
\]
\end{minipage}
~
\begin{minipage}{0.48\linewidth}
\vspace{40ex}
\[
\Infer{seq-$\imp$Intro}
   {
     \hspace*{33ex}  % remove this line if you're working in TeX
   }
   {\emptyctx |- (Q \AND P) \imp (Q \imp P) \True}
\]
\end{minipage}

\vfill

Question 2(b), on the next page, involves a proof.



\clearpage

\textbf{Question 2(b).}
Complete the seq-$\imp$Elim
case of the following proof.

\fontsz{10pt}{
\begin{conjecture}[Soundness of sequent calculus with respect to natural deduction]
~\\
  If $\Dee$ derives $\Gamma |- A \True$ then
  $A \True$ under the assumptions $\Gamma$.
\end{conjecture}
\begin{proof}
  By induction on $\Dee$.

  \textbf{Induction hypothesis:}
  If $\Dee'$ derives $\Gamma' |- A' \True$ and $\Dee' \prec \Dee$,
  then $A' \True$ under the assumptions $\Gamma'$.

  Consider cases of the rule concluding $\Dee$.

  \begin{itemize}
  \item Case: seq-assum

    \begin{llproof}
      \Pf{}{}{\xNDassume{x}{A \True} \in \Gamma} {By inversion}
      \Pf{}{}{A \True\text{~~~under assumptions $\Gamma$}} {By assumption $x$}
    \end{llproof}

  \item Case: seq-$\imp$Intro

    \begin{llproof}
      \eqPf{A}{A_1 \imp A_2}   {By inversion}
      \ePf{\Gamma}{A \True}  {Given}
      \ePf{\Gamma}{(A_1 \imp A_2) \True}  {By above equation}
      \ePf{\Gamma, \xNDassume{x}{A_1 \True}}{A_2 \True}  {By inversion}
      \Pf{}{}{A_2 \True\text{~~~under assumptions $\Gamma, \xNDassume{x}{A_1 \True}$}}
        {By IH}
      \Pf{}{}{(A_1 \imp A_2) \True\text{~~~under assumptions $\Gamma$}}
        {By nat-$\imp$Intro\noteassume{x}}
    \end{llproof}

  \item Case: seq-$\AND$Intro:
    We should do this case, but let's not.
    (This means you don't have to do it.)

  \item Case: seq-$\AND$Elim1

    \begin{llproof}
      \ePf{\Gamma}{A \True}  {Given}
      \ePf{\Gamma}{A \AND B \True}  {By inversion}
      \Pf{}{}{A \AND B \True\text{~~~under assumptions $\Gamma$}}
        {By IH}
      \Pf{}{}{A \True\text{~~~under assumptions $\Gamma$}}
        {By nat-$\AND$Elim1}
    \end{llproof}

  \item Case: seq-$\AND$Elim2

    \begin{llproof}
      \ePf{\Gamma}{A \True}  {Given}
      \ePf{\Gamma}{A_1 \AND A \True}  {By inversion}
      \Pf{}{}{A_1 \AND A \True\text{~~~under assumptions $\Gamma$}}
        {By IH}
      \Pf{}{}{A \True\text{~~~under assumptions $\Gamma$}}
        {By nat-$\AND$Elim2}
    \end{llproof}


  \item Case: seq-$\imp$Elim
    \textbf{[Complete this case]}

\vfill
  \end{itemize}
\end{proof}



}  % end fontsz

\clearpage

\section{Q3: Choice (25 points)}
\newcommand{\DC}[2]{\textsf{choose}({#1},\,{#2})}

Consider the language of terms from Q1, extended with
a \emph{choice operator} $\DC{M_1}{M_2}$, which uses \emph{either} $M_1$ or $M_2$,
unpredictably.
The new grammar is:

\begin{grammar}
integers & $n$ &
\\
variables & $x, y, z$ &
 \\
values & $v$
   &\bnfas&
   $x \bnfalt \lcabs{x}{N}
   $
\\
(Q3) terms  & $M, N$ &\bnfas&
      $x
      \bnfaltBRKn{variable}
      n
      \bnfaltBRKn{integer}
      \lcabs{x}{N}
      \bnfaltBRKn{abstraction}
      M\texttt(M\texttt)
      \bnfaltBRKn{application}
      \DC{M}{M}
      \BRKn{choice}
      $
\end{grammar}


The rules for the choice operator are eval-choose1 and eval-choose2.
The other rules are somewhat similar to those in Lecture 10, with different syntax.
The rule eval-app corresponds to the rule eval-call in Lecture 10.

\medskip

\judgbox{M \down v}{Term $M$ evaluates to $v$}
\vspace*{-1.5ex}
\begin{mathpar}
  \Infer{eval-value}
     {}
     {v \down v}
  \and
  \Infer{eval-app}
     {
       M_1 \down \lcabs{x}{N}
       \\
       M_2 \down v_2
       \\{}
       [v_2/x]N \down v
     }
     {
       \lcapp{M_1}{M_2} \down v
     }
  \and
  \Infer{eval-choose1}
     {M_1 \down v_1}
     {\DC{M_1}{M_2} \down v_1}
  \and
  \Infer{eval-choose2}
     {M_2 \down v_2}
     {\DC{M_1}{M_2} \down v_2}
\end{mathpar}

\vspace*{2ex}

Give a counterexample to the following conjecture:

\begin{conjecture}
  If $M \down v$ and $N \down v$ then $M = N$.
\end{conjecture}

Counterexample:

\bigskip

$M =
$ \bigskip

$N =
$\bigskip

$v =
$

\vfill

\textbf{Question 3(b) on next page}

\clearpage

\textbf{Question 3(b).}
A useful property of (some) programming languages is \emph{type preservation}:
if a program gives an answer, the answer should have the same type as the program.
For example, if $\DC{1}{2}$ has type $\Int$, and $\DC{1}{2} \down v$, then
$v$ should also have type $\Int$.

Giving a complete set of typing rules would take us past the lecture notes covered
in class so far.  However, this question uses a drastically simplified type system,
with only two
types: $\Int$, the type of integers (like $n$), and $\Int -> \Int$, the type of the
identity function $\lcabs{x}{x}$ (which corresponds to $\lamexp{x}{x}$ in the
language of lec10.pdf).

\begin{grammar}
 types & $S, T$
   &\bnfas&
   $\Int
   \bnfalt
   \Int -> \Int
   $
\\[1ex]
typing contexts & $\Gamma$
&\bnfas&
$
\emptyctx
\bnfaltBRKn{empty context}
\Gamma, x : S
\BRKn{$x$ has type $S$}
$
\end{grammar}

The typing rules for this question are:

\medskip

\judgbox{\Gamma |- M : S}{term $M$ has type $S$}
\begin{mathpar}
  \Infer{type-int}
      {}
      {\Gamma |- n : \Int}
  \and~~
  \Infer{type-identity}
      {
      }
      {\Gamma |- \lcabs{x}{x} : \Int -> \Int}
  \and~~
  \Infer{type-app}
      {
        \Gamma |- M_1 : S -> T
        \\
        \Gamma |- M_2 : S
      }
      {\Gamma |- \lcapp{M_1}{M_2} : T}
  \\
  \Infer{type-choose}
      {
        \Gamma |- M_1 : S
        \\
        \Gamma |- M_2 : S
      }
      {
        \Gamma |- \DC{M_1}{M_2} : S
      }
\end{mathpar}

There are many, many terms that should be typable but aren't,
such as $\lcabs{x}{3}$, but we ignore them in this question.



\textbf{3(b) part 1.} 
Complete the following derivations:

\vspace{5ex}
\[
\Infer{type-choose}
     {
       \hspace*{200pt}
     }
     {
       \emptyctx |- \DC{4}{5} : \Int
     }
\]
\vspace{12ex}
\[
\Infer{type-choose}
     {
       \hspace*{400pt}
     }
     {
       \emptyctx |-        \DC{\lcabs{x}{x}}{\lcabs{y}{y}} : \Int -> \Int
     }
\]

\vfill

\textbf{Question 3(b) part 2 on next page}
\clearpage

\textbf{3(b) part 2.}
Complete one case of type preservation: the case for eval-choose2.

Proving some of the other cases would require additional lemmas, such as a substitution lemma,
but the case for eval-choose2 should not need any lemmas.


\textbf{Hint:} Within your case for eval-choose2,
you will need to think about which rule can conclude $\emptyctx |- M : S$.

\begin{conjecture}
  If $\emptyctx |- M : S$
  and $\Dee$ derives $M \down v$
  then
  $\emptyctx |- v : S$.
\end{conjecture}
\begin{proof}
  By induction on the derivation $\Dee$.

  The induction hypothesis is:
  If $\Dee' \prec \Dee$ and $\emptyctx |- M' : S'$
  and $\Dee'$ derives $M' \down v'$
  then $\emptyctx |- v' : S'$.

  Consider cases of the rule concluding $\Dee$.

  \begin{itemize}
  \item Case for eval-choose2:

\vfill
  \end{itemize}
\end{proof}


\textbf{3(b) part 3.}
The statement of type preservation contains several objects that we could potentially induct on.
I chose to induct on the derivation of $M \down v$.
Considering only the eval-choose2 case, would inducting on $S$
have worked?  That is, the IH would require $S' \prec S$, instead of $\Dee' \prec \Dee$.
(For example, $\Int \prec \Int -> \Int$, but $\Int -> \Int \not\prec \Int -> \Int$.)

\vspace*{15ex}

\end{document}




% Local Variables:
% TeX-master: "midterm"
% End:

